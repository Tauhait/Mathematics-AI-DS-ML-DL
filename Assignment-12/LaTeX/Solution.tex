\documentclass[journal,12pt,twocolumn]{IEEEtran}
\usepackage{setspace}
\usepackage{gensymb}
\singlespacing
\usepackage[cmex10]{amsmath}

\usepackage{amsthm}
\usepackage{hyperref}
\hypersetup{
    colorlinks=true,
    linkcolor=blue,
    filecolor=magenta,      
    urlcolor=cyan,
}

\urlstyle{same}
\usepackage{mathrsfs}
\usepackage{txfonts}
\usepackage{stfloats}
\usepackage{bm}
\usepackage{cite}
\usepackage{cases}
\usepackage{subfig}

\usepackage{longtable}
\usepackage{multirow}

\usepackage{enumitem}
\usepackage{mathtools}
\usepackage{steinmetz}
\usepackage{tikz}
\usepackage{circuitikz}
\usepackage{verbatim}
\usepackage{tfrupee}
\usepackage[breaklinks=true]{hyperref}
\usepackage{graphicx}
\usepackage{tkz-euclide}
\usetikzlibrary{shapes,backgrounds}
\usepackage{verbatim}
\usetikzlibrary{calc,math}
\usepackage{listings}
    \usepackage{color}                                            %%
    \usepackage{array}                                            %%
    \usepackage{longtable}                                        %%
    \usepackage{calc}                                             %%
    \usepackage{multirow}                                         %%
    \usepackage{hhline}                                           %%
    \usepackage{ifthen}                                           %%
    \usepackage{lscape}     
\usepackage{multicol}
\usepackage{chngcntr}
\usepackage{mdframed}
\DeclareMathOperator*{\Res}{Res}

\renewcommand\thesection{\arabic{section}}
\renewcommand\thesubsection{\thesection.\arabic{subsection}}
\renewcommand\thesubsubsection{\thesubsection.\arabic{subsubsection}}

\renewcommand\thesectiondis{\arabic{section}}
\renewcommand\thesubsectiondis{\thesectiondis.\arabic{subsection}}
\renewcommand\thesubsubsectiondis{\thesubsectiondis.\arabic{subsubsection}}


\hyphenation{op-tical net-works semi-conduc-tor}
\def\inputGnumericTable{}                                 %%

\lstset{
%language=C,
frame=single, 
breaklines=true,
columns=fullflexible
}

\usepackage{chngcntr}
\counterwithin{figure}{section}

\title{AI5002}
\author{TUHIN DUTTA}
\date{January 2021}

\begin{document}
\newtheorem{theorem}{Theorem}[section]
\newtheorem{problem}{Problem}
\newtheorem{proposition}{Proposition}[section]
\newtheorem{lemma}{Lemma}[section]
\newtheorem{corollary}[theorem]{Corollary}
\newtheorem{example}{Example}[section]
\newtheorem{definition}[problem]{Definition}

\newcommand{\BEQA}{\begin{eqnarray}}
\newcommand{\EEQA}{\end{eqnarray}}
\newcommand{\define}{\stackrel{\triangle}{=}}
\bibliographystyle{IEEEtran}
\raggedbottom
\setlength{\parindent}{0pt}
\providecommand{\mbf}{\mathbf}
\providecommand{\pr}[1]{\ensuremath{\Pr\left(#1\right)}}
\providecommand{\qfunc}[1]{\ensuremath{Q\left(#1\right)}}
\providecommand{\sbrak}[1]{\ensuremath{{}\left[#1\right]}}
\providecommand{\lsbrak}[1]{\ensuremath{{}\left[#1\right.}}
\providecommand{\rsbrak}[1]{\ensuremath{{}\left.#1\right]}}
\providecommand{\brak}[1]{\ensuremath{\left(#1\right)}}
\providecommand{\lbrak}[1]{\ensuremath{\left(#1\right.}}
\providecommand{\rbrak}[1]{\ensuremath{\left.#1\right)}}
\providecommand{\cbrak}[1]{\ensuremath{\left\{#1\right\}}}
\providecommand{\lcbrak}[1]{\ensuremath{\left\{#1\right.}}
\providecommand{\rcbrak}[1]{\ensuremath{\left.#1\right\}}}
\theoremstyle{remark}
\newtheorem{rem}{Remark}
\newcommand{\sgn}{\mathop{\mathrm{sgn}}}

\providecommand{\res}[1]{\Res\displaylimits_{#1}} 

%\providecommand{\norm}[1]{\lVert#1\rVert}
\providecommand{\mtx}[1]{\mathbf{#1}}
\providecommand{\fourier}{\overset{\mathcal{F}}{ \rightleftharpoons}}
%\providecommand{\hilbert}{\overset{\mathcal{H}}{ \rightleftharpoons}}
\providecommand{\system}{\overset{\mathcal{H}}{ \longleftrightarrow}}
	%\newcommand{\solution}[2]{\textbf{Solution:}{#1}}
\newcommand{\solution}{\noindent \textbf{Solution: }}
\newcommand{\cosec}{\,\text{cosec}\,}
\providecommand{\dec}[2]{\ensuremath{\overset{#1}{\underset{#2}{\gtrless}}}}
\newcommand{\myvec}[1]{\ensuremath{\begin{pmatrix}#1\end{pmatrix}}}
\newcommand{\mydet}[1]{\ensuremath{\begin{vmatrix}#1\end{vmatrix}}}
\numberwithin{equation}{subsection}
\makeatletter
\@addtoreset{figure}{problem}
\makeatother
\let\StandardTheFigure\thefigure
\let\vec\mathbf
\renewcommand{\thefigure}{\theproblem}
\def\putbox#1#2#3{\makebox[0in][l]{\makebox[#1][l]{}\raisebox{\baselineskip}[0in][0in]{\raisebox{#2}[0in][0in]{#3}}}}
     \def\rightbox#1{\makebox[0in][r]{#1}}
     \def\centbox#1{\makebox[0in]{#1}}
     \def\topbox#1{\raisebox{-\baselineskip}[0in][0in]{#1}}
     \def\midbox#1{\raisebox{-0.5\baselineskip}[0in][0in]{#1}}
\vspace{3cm}
\title{AI5002 - Assignment 12}
\author{Tuhin Dutta\\ ai21mtech02002}
\maketitle
\newpage
\bigskip
\renewcommand{\thefigure}{\theenumi}
\renewcommand{\thetable}{\theenumi}
\begin{mdframed}
Download LaTeX from below hyperlink\\
1. \href{https://github.com/Tauhait/AI5002/tree/main/Assignment-12/LaTeX}{LaTeX}
\end{mdframed}
\subsection*{\boldsymbol{Problem\ GATE12}}
P and Q are considering to apply for a job. The probability that P applies for the job is $\dfrac{1}{4}$, the probability that P applies for the job given that Q applies for the job is $\dfrac{1}{2}$, and the probability that Q applies for the job given that P applies for the job is $\dfrac{1}{3}$. Then the probability that P does not apply for the job given that Q does not apply for the job is\\
\\
(\text{A})\ \dfrac{4}{5}\ \ \ \ \ \ (\text{B})\ \dfrac{5}{6}\ \ \ \ \ \ (\text{C})\ \dfrac{7}{8}\ \ \ \ \ \ (\text{D})\ \dfrac{11}{12}\ \ \ \ \ \
\subsection*{\boldsymbol{Solution}}
Let us define two r.v. X $\in$ $\{ 0, 1 \}$ and Y $\in$ $\{ 0, 1 \}$ representing P and Q respectively.\\

We further define the values taken by the r.v.s and their corresponding meaning,\\
$X = 0\ or\ Y = 0$ represents that P or Q does not apply for a job.\\
$X = 1\ or\ Y = 1$ represents that P or Q apply for a job.\\
\\Given, 
\begin{align}\tag{1.0}
    \begin{split}
        \pr{X=0}\ &=\ \dfrac{3}{4},\\
        \pr{X=1}\ &=\ \dfrac{1}{4},\\
        \pr{X=1\ |\ Y=1}\ &=\ \dfrac{1}{2},\ \ \ and\\
        \pr{Y=1\ |\ X=1}\ &=\ \dfrac{1}{3}.
    \end{split}
\end{align}
The probability that P does not apply for the job given that Q does not apply for the job is given by -\\
\begin{align}\tag{1.1}
    \pr{X=0\ |\ Y=0}\ &= \dfrac{\pr{(X=0) \cap (Y=0)}}{\pr{Y=0}}
\end{align}
Also, from (1.0) we can write,\\
\begin{align}\tag{1.2}
    \begin{split}
        \pr{Y=1\ |\ X=1}\ &=\ \dfrac{\pr{(X=1) \cap (Y=1)}}{\pr{X=1}}\\
        \implies
        \dfrac{1}{3}\ &=\ \dfrac{\pr{(X=1) \cap (Y=1)}}{\dfrac{1}{4}}\\
        \implies
        \dfrac{1}{12}\ &=\ \pr{(X=1) \cap (Y=1)}
    \end{split}
\end{align}
Similarly we can write,\\
\begin{align}\tag{1.3}
    \begin{split}
        \pr{X=1\ |\ Y=1}\ &=\ \dfrac{\pr{(X=1) \cap (Y=1)}}{\pr{Y=1}}\\
        \implies
        \dfrac{1}{2}\ &=\ \dfrac{\dfrac{1}{12}}{\pr{Y=1}}\\
        \implies
        \dfrac{1}{6}\ &=\ \pr{Y=1}
    \end{split}
\end{align}
From (1.3), we can also find,
\begin{align}\tag{1.4}
    \begin{split}
        \pr{Y=0}\ &=\ 1 - \dfrac{1}{6}\\
        \pr{Y=0}\ &=\ \dfrac{5}{6}\\
    \end{split}
\end{align}
To find (1.1) we use the below equation,
\begin{align}\tag{1.5}
    \begin{split}
        \pr{(X=0) \cap (Y=0)} &= 1 - \Big[\pr{(X=1) \cup (Y=1)}\Big]\\
        &= 1 - \Big[\pr{X=1} + \pr{Y=1}\\
        & \ \ \ \ \ \ \ - \pr{(X=1) \cap (Y=1)}\Big]\\
        &= 1 - \Big[\dfrac{1}{4} + \dfrac{1}{6} - \dfrac{1}{12}\Big]\\
        &= 1 - \dfrac{1}{3}\\
        &= \dfrac{2}{3}
    \end{split}
\end{align}
\\Using (1.5) and (1.4), we solve equation (1.1),
\begin{align*}\tag{1.6}
    \begin{split}
        \pr{X=0\ |\ Y=0}\ &= \dfrac{\pr{(X=0) \cap (Y=0)}}{\pr{Y=0}}\\
        \pr{X=0\ |\ Y=0}\ &= \dfrac{\dfrac{2}{3}}{\dfrac{5}{6}} = \dfrac{4}{5}\\
    \end{split}
\end{align*}
\end{document}
