\documentclass[journal,12pt,twocolumn]{IEEEtran}
\usepackage{setspace}
\usepackage{gensymb}
\singlespacing
\usepackage[cmex10]{amsmath}

\usepackage{amsthm}
\usepackage{hyperref}
\hypersetup{
    colorlinks=true,
    linkcolor=blue,
    filecolor=magenta,      
    urlcolor=cyan,
}

\urlstyle{same}
\usepackage{mathrsfs}
\usepackage{txfonts}
\usepackage{stfloats}
\usepackage{bm}
\usepackage{cite}
\usepackage{cases}
\usepackage{subfig}

\usepackage{longtable}
\usepackage{multirow}

\usepackage{enumitem}
\usepackage{mathtools}
\usepackage{steinmetz}
\usepackage{tikz}
\usepackage{circuitikz}
\usepackage{verbatim}
\usepackage{tfrupee}
\usepackage[breaklinks=true]{hyperref}
\usepackage{graphicx}
\usepackage{tkz-euclide}

\usetikzlibrary{calc,math}
\usepackage{listings}
    \usepackage{color}                                            %%
    \usepackage{array}                                            %%
    \usepackage{longtable}                                        %%
    \usepackage{calc}                                             %%
    \usepackage{multirow}                                         %%
    \usepackage{hhline}                                           %%
    \usepackage{ifthen}                                           %%
    \usepackage{lscape}     
\usepackage{multicol}
\usepackage{chngcntr}
\usepackage{mdframed}
\DeclareMathOperator*{\Res}{Res}
\usepackage{amsmath}
\renewcommand\thesection{\arabic{section}}
\renewcommand\thesubsection{\thesection.\arabic{subsection}}
\renewcommand\thesubsubsection{\thesubsection.\arabic{subsubsection}}

\renewcommand\thesectiondis{\arabic{section}}
\renewcommand\thesubsectiondis{\thesectiondis.\arabic{subsection}}
\renewcommand\thesubsubsectiondis{\thesubsectiondis.\arabic{subsubsection}}


\hyphenation{op-tical net-works semi-conduc-tor}
\def\inputGnumericTable{}                                 %%

\lstset{
%language=C,
frame=single, 
breaklines=true,
columns=fullflexible
}

\usepackage{chngcntr}
\counterwithin{figure}{section}

\title{AI5002}
\author{TUHIN DUTTA}
\date{January 2021}

\begin{document}
\newtheorem{theorem}{Theorem}[section]
\newtheorem{problem}{Problem}
\newtheorem{proposition}{Proposition}[section]
\newtheorem{lemma}{Lemma}[section]
\newtheorem{corollary}[theorem]{Corollary}
\newtheorem{example}{Example}[section]
\newtheorem{definition}[problem]{Definition}

\newcommand{\BEQA}{\begin{eqnarray}}
\newcommand{\EEQA}{\end{eqnarray}}
\newcommand{\define}{\stackrel{\triangle}{=}}
\bibliographystyle{IEEEtran}
\raggedbottom
\setlength{\parindent}{0pt}
\providecommand{\mbf}{\mathbf}
\providecommand{\pr}[1]{\ensuremath{\Pr\left(#1\right)}}
\providecommand{\qfunc}[1]{\ensuremath{Q\left(#1\right)}}
\providecommand{\sbrak}[1]{\ensuremath{{}\left[#1\right]}}
\providecommand{\lsbrak}[1]{\ensuremath{{}\left[#1\right.}}
\providecommand{\rsbrak}[1]{\ensuremath{{}\left.#1\right]}}
\providecommand{\brak}[1]{\ensuremath{\left(#1\right)}}
\providecommand{\lbrak}[1]{\ensuremath{\left(#1\right.}}
\providecommand{\rbrak}[1]{\ensuremath{\left.#1\right)}}
\providecommand{\cbrak}[1]{\ensuremath{\left\{#1\right\}}}
\providecommand{\lcbrak}[1]{\ensuremath{\left\{#1\right.}}
\providecommand{\rcbrak}[1]{\ensuremath{\left.#1\right\}}}
\theoremstyle{remark}
\newtheorem{rem}{Remark}
\newcommand{\sgn}{\mathop{\mathrm{sgn}}}

\providecommand{\res}[1]{\Res\displaylimits_{#1}} 

%\providecommand{\norm}[1]{\lVert#1\rVert}
\providecommand{\mtx}[1]{\mathbf{#1}}
\providecommand{\fourier}{\overset{\mathcal{F}}{ \rightleftharpoons}}
%\providecommand{\hilbert}{\overset{\mathcal{H}}{ \rightleftharpoons}}
\providecommand{\system}{\overset{\mathcal{H}}{ \longleftrightarrow}}
	%\newcommand{\solution}[2]{\textbf{Solution:}{#1}}
\newcommand{\solution}{\noindent \textbf{Solution: }}
\newcommand{\cosec}{\,\text{cosec}\,}
\providecommand{\dec}[2]{\ensuremath{\overset{#1}{\underset{#2}{\gtrless}}}}
\newcommand{\myvec}[1]{\ensuremath{\begin{pmatrix}#1\end{pmatrix}}}
\newcommand{\mydet}[1]{\ensuremath{\begin{vmatrix}#1\end{vmatrix}}}
\numberwithin{equation}{subsection}
\makeatletter
\@addtoreset{figure}{problem}
\makeatother
\let\StandardTheFigure\thefigure
\let\vec\mathbf
\renewcommand{\thefigure}{\theproblem}
\def\putbox#1#2#3{\makebox[0in][l]{\makebox[#1][l]{}\raisebox{\baselineskip}[0in][0in]{\raisebox{#2}[0in][0in]{#3}}}}
     \def\rightbox#1{\makebox[0in][r]{#1}}
     \def\centbox#1{\makebox[0in]{#1}}
     \def\topbox#1{\raisebox{-\baselineskip}[0in][0in]{#1}}
     \def\midbox#1{\raisebox{-0.5\baselineskip}[0in][0in]{#1}}
\vspace{3cm}
\title{AI5002 - Assignment 1}
\author{Tuhin Dutta\\ ai21mtech02002}
\maketitle
\newpage
\bigskip
\renewcommand{\thefigure}{\theenumi}
\renewcommand{\thetable}{\theenumi}
\begin{mdframed}
Download code and LaTeX from below hyperlinks\\
1. \href{https://github.com/Tauhait/AI5002/Assignment-1/Code}{Code}


2. \href{https://github.com/Tauhait/AI5002/Assignment-1/LaTeX}{LaTeX}
\end{mdframed}
\subsection*{\boldsymbol{Problem}}

6.1.5 Find an expression for $p_A(x)$ using the definition.\\

\subsection*{\boldsymbol{Solution}}\\
Given $X_1$\ \sim N(0,1),\ $X_2$ \sim\ N(0,1),\ and\ $V = X_1^2 + X_2^2$.\\

So we can write A as\\
A = \sqrt{V}\\
or,\ A = \sqrt{X_1^2 + X_2^2}\\

Since joint distribution is not mentioned so we assume $X_1\ and\ X_2$ to be independent otherwise the distribution of A would be unknown.\\

By definition, the distribution of A is Chi with two degrees of freedom or Rayleigh.\\

The CDF given by $F_A(x)$\ can\ be\ written\ as\\
\\
= P (A \leq x )\\
\\
= P (\sqrt{X_1^2\ +\ X_2^2} \leq x)\\
\\
= \iint\displaylimits_{\sqrt{X_1^2\ +\ X_2^2}\ \leq\ x}^{} F(x_1, x_2)\ \,dx_1\,dx_2
\\
= \iint\displaylimits_{\sqrt{X_1^2\ +\ X_2^2}\ \leq\ x}^{} F(x_1) . F(x_2)\ \,dx_1\,dx_2 
\\
= \iint\displaylimits_{\sqrt{X_1^2\ +\ X_2^2}\ \leq\ x}^{} \frac{1}{\sqrt{2\pi\sigma^2}}.\  \frac{1}{\sqrt{2\pi\sigma^2}}.\ e^{\frac{-x_1^2}{2\sigma^2}}.\ e^{\frac{-x_2^2}{2\sigma^2}}\ \,dx_1\,dx_2\\
= \iint\displaylimits_{\sqrt{X_1^2\ +\ X_2^2}\ \leq\ x}^{} \frac{1}{2\pi\sigma^2}.\ e^{\frac{-(x_1^2\ +\ x_2^2)}{2\sigma^2}}\ \,dx_1\,dx_2\ \ \ \ \ \ \ \ \ \ \ (1)\\

Now, we perform a transformation of variables to simplify solution.\\
Let x_1 = rcos\ \theta\ and,\ x_2 = rsin\ \theta.\\

\implies \sqrt{x_1^2+x_2^2} = r\\

Also, \theta=\tan^{-1}{\frac{x_2}{x_1}}\\

\\Since we're using transformation of two variables, we use Jacobian matrix here\\

\mathbf{J}_{x_1,x_2} =
\begin{bmatrix}
  \frac{\partial x_1}{\partial r} & 
    \frac{\partial x_1}{\partial \theta} \\
    \\
    \frac{\partial x_2}{\partial r} & 
  \frac{\partial x_2}{\partial \theta}
\end{bmatrix}
= 
\begin{bmatrix}
  \cos \theta & -r\sin \theta \\ 
    \sin \theta & r\cos \theta \\
\end{bmatrix}
\\
\\
\det(\mathbf{J}_{x_1,x_2})= r\ \cos^2 \theta + r\ \sin^2 \theta\ = r\\

Thus, F(r, \theta) = r.F(x_1, x_2)\\
\ where\ (x_1, x_2)\ are\ written\ in\ terms\ of\ (r, \theta).\\

Now, we can write (1) as\\
$$
\iint\displaylimits_{r\ \leq\ x}^{} \frac{1}{2\pi\sigma^2} .\ e^{\frac{-r^2}{2\sigma^2}}.\ r\,dr\,d\theta\\
$$
or,
$$
 \int\limits_{\theta=0}^{2\pi} \int\limits_{r=0}^{x}\ \frac{1}{2\pi\sigma^2} .\ e^{\frac{-r^2}{2\sigma^2}}.\ r\,dr\,d\theta\ \ \ \ \ \ \ \ \ \ \ (2)\\
$$
\begin{equation*}
\begin{split}
Let\ u\ =\ -\frac{r^2}{2\sigma^2}\\
\,du\ =\ -\frac{2r}{2\sigma^2} \,dr\\
r\,dr\ =\ -\sigma^2 \,du\\
\\where\ 
0\leq\ r\ \leq\ x\ and\ 
0\leq\ u\ \leq\ -\frac{x^2}{2\sigma^2}
\end{split}
\end{equation*}\\
Substituting u and r\,dr in (2), we get\\
$$
 \int\limits_{\theta=0}^{2\pi} \int\limits_{u=0}^{-\frac{x^2}{2\sigma^2}}\ \frac{1}{2\pi\sigma^2} .\ e^{u}.\ (-\sigma^2)\ \,du\,d\theta\\
$$
\[
= \int\limits_{\theta=0}^{2\pi}-\frac{1}{2\pi} \int\limits_{u=0}^{-\frac{x^2}{2\sigma^2}}\ e^{u}\ \,du\,d\theta
= -\frac{1}{2\pi}\int\limits_{\theta=0}^{2\pi}\ (e^{-\frac{x^2}{2\sigma^2}}} - 1)\ \,d\theta
\]
= 1\ -\ \exp\ ({-{x^2}}/{2\sigma^2})
\\
\begin{mdframed}
Thus PDf of the above CDF expression is derived as\\

p_A(x) = \frac{d}{dx} F_A(x)\\

= - e^{-\frac{x^2}{2\sigma^2}}.\ (-\frac{2x}{2\sigma^2})\\

= \large\boldsymbol{e^{-\frac{x^2}{2\sigma^2}}\ .\ \frac{x}{\sigma^2}}\\
\end{mdframed}
\end{document}