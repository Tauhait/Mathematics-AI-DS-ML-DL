\documentclass[journal,12pt,twocolumn]{IEEEtran}
\usepackage{setspace}
\usepackage{gensymb}
\singlespacing
\usepackage[cmex10]{amsmath}

\usepackage{amsthm}
\usepackage{hyperref}
\hypersetup{
    colorlinks=true,
    linkcolor=blue,
    filecolor=magenta,      
    urlcolor=cyan,
}

\urlstyle{same}
\usepackage{mathrsfs}
\usepackage{txfonts}
\usepackage{stfloats}
\usepackage{bm}
\usepackage{cite}
\usepackage{cases}
\usepackage{subfig}

\usepackage{longtable}
\usepackage{multirow}

\usepackage{enumitem}
\usepackage{mathtools}
\usepackage{steinmetz}
\usepackage{tikz}
\usepackage{circuitikz}
\usepackage{verbatim}
\usepackage{tfrupee}
\usepackage[breaklinks=true]{hyperref}
\usepackage{graphicx}
\usepackage{tkz-euclide}
\usetikzlibrary{shapes,backgrounds}
\usepackage{verbatim}
\usetikzlibrary{calc,math}
\usepackage{listings}
    \usepackage{color}                                            %%
    \usepackage{array}                                            %%
    \usepackage{longtable}                                        %%
    \usepackage{calc}                                             %%
    \usepackage{multirow}                                         %%
    \usepackage{hhline}                                           %%
    \usepackage{ifthen}                                           %%
    \usepackage{lscape}     
\usepackage{multicol}
\usepackage{chngcntr}
\usepackage{mdframed}
\DeclareMathOperator*{\Res}{Res}

\renewcommand\thesection{\arabic{section}}
\renewcommand\thesubsection{\thesection.\arabic{subsection}}
\renewcommand\thesubsubsection{\thesubsection.\arabic{subsubsection}}

\renewcommand\thesectiondis{\arabic{section}}
\renewcommand\thesubsectiondis{\thesectiondis.\arabic{subsection}}
\renewcommand\thesubsubsectiondis{\thesubsectiondis.\arabic{subsubsection}}


\hyphenation{op-tical net-works semi-conduc-tor}
\def\inputGnumericTable{}                                 %%

\lstset{
%language=C,
frame=single, 
breaklines=true,
columns=fullflexible
}

\usepackage{chngcntr}
\counterwithin{figure}{section}

\title{AI5002}
\author{TUHIN DUTTA}
\date{January 2021}

\begin{document}
\newtheorem{theorem}{Theorem}[section]
\newtheorem{problem}{Problem}
\newtheorem{proposition}{Proposition}[section]
\newtheorem{lemma}{Lemma}[section]
\newtheorem{corollary}[theorem]{Corollary}
\newtheorem{example}{Example}[section]
\newtheorem{definition}[problem]{Definition}

\newcommand{\BEQA}{\begin{eqnarray}}
\newcommand{\EEQA}{\end{eqnarray}}
\newcommand{\define}{\stackrel{\triangle}{=}}
\bibliographystyle{IEEEtran}
\raggedbottom
\setlength{\parindent}{0pt}
\providecommand{\mbf}{\mathbf}
\providecommand{\pr}[1]{\ensuremath{\Pr\left(#1\right)}}
\providecommand{\qfunc}[1]{\ensuremath{Q\left(#1\right)}}
\providecommand{\sbrak}[1]{\ensuremath{{}\left[#1\right]}}
\providecommand{\lsbrak}[1]{\ensuremath{{}\left[#1\right.}}
\providecommand{\rsbrak}[1]{\ensuremath{{}\left.#1\right]}}
\providecommand{\brak}[1]{\ensuremath{\left(#1\right)}}
\providecommand{\lbrak}[1]{\ensuremath{\left(#1\right.}}
\providecommand{\rbrak}[1]{\ensuremath{\left.#1\right)}}
\providecommand{\cbrak}[1]{\ensuremath{\left\{#1\right\}}}
\providecommand{\lcbrak}[1]{\ensuremath{\left\{#1\right.}}
\providecommand{\rcbrak}[1]{\ensuremath{\left.#1\right\}}}
\theoremstyle{remark}
\newtheorem{rem}{Remark}
\newcommand{\sgn}{\mathop{\mathrm{sgn}}}

\providecommand{\res}[1]{\Res\displaylimits_{#1}} 

%\providecommand{\norm}[1]{\lVert#1\rVert}
\providecommand{\mtx}[1]{\mathbf{#1}}
\providecommand{\fourier}{\overset{\mathcal{F}}{ \rightleftharpoons}}
%\providecommand{\hilbert}{\overset{\mathcal{H}}{ \rightleftharpoons}}
\providecommand{\system}{\overset{\mathcal{H}}{ \longleftrightarrow}}
	%\newcommand{\solution}[2]{\textbf{Solution:}{#1}}
\newcommand{\solution}{\noindent \textbf{Solution: }}
\newcommand{\cosec}{\,\text{cosec}\,}
\providecommand{\dec}[2]{\ensuremath{\overset{#1}{\underset{#2}{\gtrless}}}}
\newcommand{\myvec}[1]{\ensuremath{\begin{pmatrix}#1\end{pmatrix}}}
\newcommand{\mydet}[1]{\ensuremath{\begin{vmatrix}#1\end{vmatrix}}}
\numberwithin{equation}{subsection}
\makeatletter
\@addtoreset{figure}{problem}
\makeatother
\let\StandardTheFigure\thefigure
\let\vec\mathbf
\renewcommand{\thefigure}{\theproblem}
\def\putbox#1#2#3{\makebox[0in][l]{\makebox[#1][l]{}\raisebox{\baselineskip}[0in][0in]{\raisebox{#2}[0in][0in]{#3}}}}
     \def\rightbox#1{\makebox[0in][r]{#1}}
     \def\centbox#1{\makebox[0in]{#1}}
     \def\topbox#1{\raisebox{-\baselineskip}[0in][0in]{#1}}
     \def\midbox#1{\raisebox{-0.5\baselineskip}[0in][0in]{#1}}
\vspace{3cm}
\title{AI5002 - Assignment 9}
\author{Tuhin Dutta\\ ai21mtech02002}
\maketitle
\newpage
\bigskip
\renewcommand{\thefigure}{\theenumi}
\renewcommand{\thetable}{\theenumi}
\begin{mdframed}
Download code and LaTeX from below hyperlinks\\
1. \href{https://github.com/Tauhait/AI5002/blob/main/Assignment-9/Code/AxiomProb\_6\_20.py}{Code/AxiomProb\_6\_20.py}


2. \href{https://github.com/Tauhait/AI5002/tree/main/Assignment-9/LaTeX}{LaTeX}
\end{mdframed}
\subsection*{\boldsymbol{Problem\ 6.20}}
\begin{flushleft} An unbiased die is thrown twice. Let the event
A be `odd number on the first throw' and B
be event `odd number on the second throw'.
Check the independence of the events A and
B. \end{flushleft}

\subsection*{\boldsymbol{Solution}}\\

We know two events are said to be independent if P(A $\mathop{\cap}$ B) = P(A).P(B)\\
\\
Let us define the random variable S as `Throwing a unbiased dice twice'.
The sample space of random variable S is given by -\\
$
S = \{\\
(1,1)\ (1,2)\ (1,3)\ (1,4)\ (1,5)\ (1,6),\\
(2,1)\ (2,2)\ (2,3)\ (2,4)\ (2,5)\ (2,6),\\
(3,1)\ (3,2)\ (3,3)\ (3,4)\ (3,5)\ (3,6),\\
(4,1)\ (4,2)\ (4,3)\ (4,4)\ (4,5)\ (4,6),\\
(5,1)\ (5,2)\ (5,3)\ (5,4)\ (5,5)\ (5,6),\\
(6,1)\ (6,2)\ (6,3)\ (6,4)\ (6,5)\ (6,6)  \}$
\\
\\
We define two random variables X and Y where X denotes `Throwing an odd number on first throw' and Y denotes `Throwing an odd number on second throw'.

The sample space of X is given by - \\
$X = \{\\
(1,1)\ (1,2)\ (1,3)\ (1,4)\ (1,5)\ (1,6),\\
(3,1)\ (3,2)\ (3,3)\ (3,4)\ (3,5)\ (3,6),\\
(5,1)\ (5,2)\ (5,3)\ (5,4)\ (5,5)\ (5,6)  \} $

Probability of odd number on the first throw\\
\begin{align}
    P(X) = \frac{n(X)}{n(S)} = \frac{18}{36} = \frac{1}{2}
\end{align}

The sample space of random variable Y is given by - \\
$Y = \{\\
(1,1)\ (1,3)\ (1,5),\\
(2,1)\ (2,3)\ (2,5),\\
(3,1)\ (3,3)\ (3,5),\\
(4,1)\ (4,3)\ (4,5),\\
(5,1)\ (5,3)\ (5,5),\\
(6,1)\ (6,3)\ (6,5)  \} $

Probability of odd number on the second throw\\
\begin{align}
    P(Y) = \frac{n(Y)}{n(S)} = \frac{18}{36} = \frac{1}{2}
\end{align}

We define one more random variable Z which is defined as `Throwing odd numbers on both first and second throws' and the sample space is given by - \\
$
Z = \{\\
(1,1)\ (1,3)\ (1,5),\\
(3,1)\ (3,3)\ (3,5),\\
(5,1)\ (5,3)\ (5,5) \}$

Probability of odd number on the first and second throw.\\
\begin{align}
    P(Z) = \frac{n(Z)}{n(S)} = \dfrac{9}{36} = \dfrac{1}{4}
\end{align}

Also, P(X) . P(Y) = \dfrac{1}{2} . \dfrac{1}{2} = \dfrac{1}{4}\\

Since,\\ P(Z) = P(X) . P(Y)\\
Therefore, X and Y are independent random variables.
\end{document}
